\section{Introduction}
Attention, il vous est nécessaire d’avoir des bases en HTML et JavaScript pour
suivre ce TP. Si ce n’est pas le cas, je vous conseille d’essayer de faire le TP
Web galerie avant celui-ci.
Dans ce TP vous allez créer une horloge, que vous pourrez afficher sur des pages
html.
Ce TP va vous apprendre à insérer du code JavaScript dans
n’importe\footnote{Certaines pages empêchent l’exécution de JavaScript de cette
manière, mais cela reste plus une exception qu’une règle.} quelle page, et aussi
dessiner sur des canvas. Vous verrez aussi un tout petit peu de trigonométrie,
mais rien de trop compliqué !

\section{HTML d’exemple}
Pour tester notre code, nous allons avoir besoin d’une page de base sur laquelle
effectuer nos tests.
\paragraph{}
Pour cela nous allons utiliser le code html suivant :

\begin{minted}{html}
<!DOCTYPE HTML>
<html>
    <head>
        <meta charset="utf-8">
        <title>Page de démo</title>
    </head>
    <body>
        <h1>Bonjour.</h1>
        <script src="js/script.js"></script>
    </body>
</html>
\end{minted}

Ce code est suffisant pour charger du JavaScript, donc nous n’allons pas aller plus loin.

\section{Insertion d’un canvas dans la page en javascript}
\subsection{HTML}
Nous allons insérer un div, contenant un canvas. Le div sera utilisé pour bouger
notre horloge, et lui dessiner une bordure, tandis que le canvas sera utilisé
pour dessiner l’horloge.
\paragraph{}
Le code HTML est le suivant :
\begin{minted}{html}
<div id="_horloge" style="top: 0; left: 0;">
    <canvas id="_horloge_canvas"></canvas>
</div>
\end{minted}

\subsection{CSS}
Nous allons insérer un petit peu de css :
\begin{minted}{css}
#_horloge {
    position: fixed;
    width: 200px;
    height: 200px;
    background: rgba(14,118,211,0.5);
    border: 1px solid rgb(14,118,211);
    border-radius: 50%;
    z-index: 10000000;
}

#_horloge_canvas {
    position: absolute;
    top: 4px;
    left: 4px;
    width: calc(100% - 8px);
    height: calc(100% - 8px);
    background: #eee;
    border-radius: 50%;
    pointer-events: none;
}
\end{minted}
Il vous sera possible de rajouter quelques propriétés pour rendre l’horloge plus
jolie, mais ce code est nécessaire.

\subsection{L’insertion}
Pour effectuer l’insertion de ces morceaux de code nous allons utiliser la
fonction JavaScript
\href{https://developer.mozilla.org/fr/docs/Web/API/Element/insertAdjacentHTML}{element.insertAdjacentHTML(position, text)}.
Cette fonction permet d’ajouter du code de manière plus efficace que
« element.innerHTML », très communément utilisée, que, du coup, nous ne verrons
pas. (Ici nous devrons donc insérer notre code dans les éléments
« document.head » pour le css, et document.body pour le div et le canvas)
\paragraph{}
Vous devrez utiliser cette fonction après l’initialisation de la page, via
l’évènement « DOMContentLoaded ».

\section{Gestion du déplacement de notre horloge}
L’horloge sera déplacée par le div. Pour cela nous allons donc devoir écouter
les évènements suivants : « mousedown », « mousemove » et « mouseup ». Les deux
évènements « mousemove » et « mouseup » devront être ajoutés sur le document,
sinon si l’on bouge trop vite la souris, il est possible de perdre le contrôle
de l’horloge.
\paragraph{}
Afin de savoir si l’évènement « mousemove » doit faire bouger l’horloge, il faut
que « mousedown » et « mouseup » activent et désactivent une classe sur
l’élement div, via son objet
\href{https://developer.mozilla.org/fr/docs/Web/API/Element/classList}{classList}.
\paragraph{}
Enfin, pour gérer le déplacement il faut modifier l’attribut « style.top » et
« style.left » du div. Il faut aussi récupérer les coordonnées actuelles de
l’élément. Pour cela vous pouvez utiliser le code suivant :
\begin{minted}{js}
let REGEX = /(-?\d+)(?:px)/;

function coordsElement(elm) {
    let resultat = {};
    resultat["x"] = parseInt(REGEX.exec(elm.style.left)[1]);
    resultat["y"] = parseInt(REGEX.exec(elm.style.top)[1]);
    return resultat;
}
\end{minted}

\section{Dessin de l’horloge}
%TODO yay trigo and lines

\section{Rafraîchissement automatique}
%TODO requestAnimationFrame

\section{Transformation en addon}
%TODO develop
1- remove all comments (or replace them with the multiline kind)
2- tout déplacer dans une fonction
3- ajouter le code nécessaire pour executer la fonction
4- créer un favoris avec pour url "javascript:votre code final"
5- ouvrir le favoris sur une page quelconque
